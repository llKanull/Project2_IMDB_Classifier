\chapter{Introduction}
This paper talks about classification problems in Machine Learning, in particular the classification of movies according to different features. We aim to get the rating of a movie based on certain information provided to us. This is useful for companies that want to predict the rating of a movie before it is released, especially for companies that invest in movies like cinemas and production houses.This can help them decide if it is worth it to bring this movie to their audience.
One of the key challenges here is to find the right features to help us determine how the movie will be received by the general population. Although there is a lot of research in developing machine learning models, an important part is to figure out how to apply them to real world use cases and this paper aims to have a discussion about this issue.
In this paper, we show how to optimise various models by mainly doing the following procedures: First, we do data preprocessing. Secondly, we construct various models by computing the hyperparameters that give the best results and selecting optimum features. The main part of the research done here is the construction of an ensemble model that is built on two different models. The ensemble model has the benefit of combination of results of two complementary models, thus helping us get the positives of several methods. We also benchmark different models by using learning curves for example, which gives us information like their fit on the dataset.
